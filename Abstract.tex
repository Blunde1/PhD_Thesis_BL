\chapter*{Abstract}

\tolerance=1
\emergencystretch=\maxdimen
\hyphenpenalty=10000
\hbadness=10000

Advanced statistical computations have become increasingly important, as with the increased flexibility of models capturing complex relationships in new data and use-cases,
comes increased difficulties of fitting procedures for the models.
For example, if the model is complex, involving multiple sources of randomness, then the probability density function used in maximum likelihood estimation typically does not have a closed form.
On the other hand, in regression type problems the closed form of the assumed conditional distribution of the response is often known. However, the relationship between features and response can be complex, high dimensional and is generally unknown, motivating non-parametric procedures that come with new sets of fitting problems.


This thesis explores techniques utilizing the local curvature of objective functions, and using the information inherent in this local curvature, to create more stable and automatic fitting procedures.
In the first paper, a saddlepoint adjusted inverse Fourier transform is proposed.
The method performs accurate numerical inversion, even in the tails of the distribution.
This allow practitioners to specify their models in terms of the characteristic function which often exists in closed form.
The second paper proposes an information criterion for the node-splits, after greedy recursive binary splitting, in gradient boosted trees.
This alleviates the need for computationally expensive cross validation and expert opinions in tuning hyperparameters associate gradient tree boosting.
The third paper focuses on the implementation of the theory presented in the second paper into the R package agtboost, and also builds on the information criterion to suggest an adjustment of ordinary greedy recursive binary splitting, better suited to gradient tree boosting.

%Adaptivity of methods are important!



%
%%Saddlepoint adjusted inversion of characteristic functions
%\insertToPaperList{I}{Lunde, Berent Ånund Strømnes, Tore Selland Kleppe, and Hans Julius Skaug (2020).}{\PaperTitleOne}{Journal of Journals}{ 57}{, 80-93.}
%
%For certain types of statistical models, the characteristic function (Fourier transform) is available in closed form, whereas the probability density function has an intractable form, typically as an infinite sum of probability weighted densities. Important such examples include solutions of stochastic differential equations with jumps, the Tweedie model, and Poisson mixture models. We propose a novel and general numerical method for retrieving the probability density function from the characteristic function. Unlike methods based on direct application of quadrature to the inverse Fourier transform, the proposed method allows accurate evaluation of the log-probability density function arbitrarily far out in the tail. Moreover, unlike saddlepoint approximations, the proposed methodology can be made arbitrarily accurate. Owing to these properties, the proposed method is computationally stable and very accurate under log-likelihood optimisation. The method is illustrated for a normal variance-mean mixture, and in an application of maximum likelihood estimation to a jump diffusion model for financial data.
%
%
%%An information criterion for automatic gradient tree boosting
%\insertToPaperList{II}{Lunde, Berent Ånund Strømnes, Tore Selland Kleppe, and Hans Julius Skaug (2020).}{\PaperTitleTwo}{To be submitted
%	for publication in Journal of Journals.}{}{}
%
%An information theoretic approach to learning the complexity of CART trees and the number of trees in gradient tree boosting is proposed.
%The optimism (test loss minus training loss) of the greedy leaf splitting procedure is shown to be the maximum of a Cox-Ingersoll-Ross process, from which a generalization-error based information criterion is formed. The proposed procedure allows fast local model selection without cross validation based hyper parameter tuning, and hence efficient and automatic comparison among the large number of models performed during each boosting iteration.
%Relative to XGBoost, speedups on numerical experiments ranges from around 10 to about 1400, at similar predictive-power measured in terms of test-loss.
%
%
%%aGTBoost: Adaptive and Automatic Gradient Tree Boosting Computations
%\insertToPaperList{III}{Lunde, Berent Ånund Strømnes, Tore Selland Kleppe, and Hans Julius Skaug (2020).}{\PaperTitleThree}{To be submitted
%	for publication in Journal of Journals.}{}{}
%
%aGTBoost is an R package implementing fast gradient 
%tree boosting computations in a manner similar to other established packages such 
%as the xgboost and LightGBM packages, but with significant decreases in computation time and 
%the mathematical and technical knowledge imposed on the user. 
%The user need only pass a response vector and a design matrix, and define
%the type of problem through the choice of loss function, selected from a pre-defined list of possibilities.
%The package automatically takes care of split no-split decisions and selects the number of trees
%in the gradient tree boosting ensemble, i.e. aGTBoost adapts the complexity of the ensemble automatically
%to the information in the data passed by the user.
%All of this is done on-line during training of the ensemble, which is made possible 
%by utilizing developments in information theory for tree algorithms -cite-.
%aGTBoost also comes with a feature importance function that calculates with respect to 
%approximate generalization loss, thus eliminating the common practice of inserting noise features, and a useful model validation function that performs the Kolmogorov-Smirnov test on the learned distribution.

